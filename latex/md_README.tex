Simulator for network connection points, paths between them and packets flowing in midst of this all. The approach of the implementation is somewhat top-\/level and statistics oriented. 



\subsection*{Usage}


\begin{DoxyItemize}
\item Build with make (Makefile included and should handle most platforms) \begin{quote}


Running the executive\+:

$>$``` $>$./ns \mbox{[}relative-\/xml-\/file-\/path\mbox{]} $>$./ns random \mbox{[}node-\/count\mbox{]} \mbox{[}edge-\/count\mbox{]} \mbox{[}packet-\/generator-\/count\mbox{]} $>$``` \end{quote}

\item Shows a gui that has few keys mapped and mouse controls. Runs a 10s elapse by default.
\item To modify the network, use xml files (resources directory). \begin{quote}


$>$Basic xml layout\+:

$>$```xml $>$$<$network$>$ $<$node address=\char`\"{}somename\char`\"{} x=\char`\"{}1.\+0\char`\"{} y=\char`\"{}1.\+0\char`\"{}$>$ $<$application type=\char`\"{}\+Packet\+Receiver\char`\"{}$>$ $<$application type=\char`\"{}\+Packet\+Generator\char`\"{}$>$ $<$destination address=\char`\"{}\+A\char`\"{}$>$ ... more destinations ...

... more applications ...

$<$/node$>$

... more nodes ...

$<$link source=\char`\"{}\+A\char`\"{} destination=\char`\"{}\+B\char`\"{} directed=\char`\"{}false\char`\"{} type=\char`\"{}\+Wireless\char`\"{} speed=\char`\"{}1.\+0\char`\"{} delay=\char`\"{}1.\+0\char`\"{} weight=\char`\"{}1.\+0\char`\"{}$>$

... more links ...

$>$$<$/network$>$ $>$```

\end{quote}

\item Applications to be used are
\begin{DoxyItemize}
\item Packet\+Receiver
\item Packet\+Generator -\/ defines the destinations for the generated packets
\item Random\+Router
\item Test\+Router
\end{DoxyItemize}
\item Routing can be done with
\begin{DoxyItemize}
\item the random applications
\item with aforementioned links (xml) between nodes.
\item few implemented path algorithms (Currently only shortest path is implemented)
\begin{DoxyItemize}
\item This requires tweeking of the code 


\end{DoxyItemize}
\end{DoxyItemize}
\end{DoxyItemize}

\subsection*{Architecture}

\subsubsection*{Network\+Simulator}


\begin{DoxyItemize}
\item Control layer for the program.
\item The Network\+Simulator inherits a Network and a Timer thus enclosing their functionality.
\end{DoxyItemize}

\paragraph*{Network}


\begin{DoxyItemize}
\item Network has function which returns pointers to all links / nodes in network
\end{DoxyItemize}

Example of getting a specific node by address\+: \begin{DoxyVerb}`network.getNode("some address")`
\end{DoxyVerb}


\paragraph*{Timer}


\begin{DoxyItemize}
\item Clocks the processes in the program.
\item Can be paused.
\end{DoxyItemize}

\subsubsection*{Node}


\begin{DoxyItemize}
\item Abstract layer for the nodes (mainly the Application\+Node)
\end{DoxyItemize}

\paragraph*{Application\+Node}


\begin{DoxyItemize}
\item An Application\+Node (subclass of Node) has a vector of Applications, which are activated for the receives and sends every clock cycle. Applications process host node\textquotesingle{}s packet queue by reference, so the order of the applications is crucial.
\end{DoxyItemize}

\subsubsection*{Application}


\begin{DoxyItemize}
\item Application serve as a base class for Application logic / in other words for Routers, Packet\+Generators (spew random packets) etc.
\end{DoxyItemize}

\paragraph*{Application\+Factory}


\begin{DoxyItemize}
\item Application\+Factory can be used for spawning concrete implementations of the Application class.
\end{DoxyItemize}

\subsubsection*{Link}


\begin{DoxyItemize}
\item Links are composed of a pair of two nodes = pair$<$\+Node $\ast$, Node $\ast$$>$.
\item The simplest link type is a directed link, but undirected or bidirectional links are implicit in current implementations. They are implemented by having two directed links.
\end{DoxyItemize}

\paragraph*{Parametric\+Link}


\begin{DoxyItemize}
\item Links have source and address. They also enclose parameters for speed, delay and weight to simulate bandwidth, latency and other such factors.
\end{DoxyItemize}

\subsection*{Testing}


\begin{DoxyItemize}
\item Uses \href{https://github.com/philsquared/Catch}{\tt Catch} testing framework
\item Basic usage is through arguments. The executive can be run normally by not giving it any parameters. Parameters override the executive to be utilized for testing. \begin{quote}


$>$Running tests for xml (linux)\+:

$>$``` $>$./ns -\/n xml $>$```

$>$./ns refers to the built executive. The parameters -\/n and xml refer to \textquotesingle{}name\textquotesingle{} and the tag of the test. To get tags for all tests use\+:

$>$``` $>$./ns -\/l $>$```

$>$and for help and usage\+:

$>$``` $>$./ns -\/h $>$```

\subsection*{}



\subsection*{Documentation}

\end{quote}

\item Included in the root folder are two doxygen configuration files.
\end{DoxyItemize}

Doxygen documentation can be generated with these configuration by running\+: \begin{DoxyVerb}`doxygen <doxygen-conf-file>`
\end{DoxyVerb}


This will create documentation in html by default of the classes and their related material

\subsubsection*{Requirements for the documentation}

Doxygen -\/ of course -\/ to generate the documentation with it. This is the only requirement, if the non-\/graphs configuration is used.

The directory ./doc might be necessary to create by hand, if doxygen doesn\textquotesingle{}t handle directory creation for you.

Graphviz should be installed to use doxygen with the graph enabled configuration. The graphs are built with graphviz\textquotesingle{}s dot tool.

\subsubsection*{Commenting for the documentation}

Doxygen understands many variations of commenting. We should use these conventions\+: \begin{DoxyVerb}/**
 * Block comments for longer comments (classes etc)
 */

/** one liners, that should be included in the doxygen */

/* one liners that are only for code reading / no doxygen support */

// same as the above (one liners)
\end{DoxyVerb}






\subsection*{Directory structure}

This repository contains three subdirectories\+:


\begin{DoxyItemize}
\item plan/ for the plan
\item doc/ for the final documentation
\item src/ for all the source code 


\end{DoxyItemize}

\subsection*{License}

M\+I\+T license applies for all that\textquotesingle{}s included here, if not explicitly stated otherwise (in files). See L\+I\+C\+E\+N\+S\+E. 



\subsection*{Team}

Agrasagar Bhattacharyya Chamran Moradi Ashour Joni Turunen Tommi Gröhn 